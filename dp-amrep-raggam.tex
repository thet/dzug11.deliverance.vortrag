\documentclass{beamer}
\usepackage[utf8]{inputenc}
\usepackage[english,ngerman]{babel}
\usepackage{beamerthemesplit}


\definecolor{fhgreen}{rgb}{0.47, 0.72, 0.00}
\definecolor{fhgrey}{rgb}{0.89, 0.89, 0.90}
%\setbeamercolor{normal text}{bg=fhgreen}

%\usecolortheme[named=fhgreen]{structure}
%\useoutertheme{infolines}
\setbeamertemplate{navigation symbols}{} %no nav symbols
\setbeamertemplate{items}[square]

\logo{\includegraphics[scale=0.15]{resources/fh-logo}}

% START
\title{Activity Model Runtime Engine for Python}
\author[J. Raggam]{Johannes Raggam}
\institute[IMA]{
  FH-Joanneum\\
  Informationsmanagement\\
  Supervisor: FH-Prof. Dipl.-Ing. Peter Salhofer\\
  External Supervisor: Jens W. Klein; BlueDynamics Alliance\\
}
\date[October 2009]{13.10.2009}



\begin{document}
\shorthandoff{"} % Let the usage of normal quotation marks

\frame{

    \titlepage

%    \begin{tabular}{c c c}
%        \includegraphics[scale=0.5]{resources/logo_ima}
%        & &
%        \includegraphics[scale=0.6]{resources/logo_bda}
%    \end{tabular}
}

% INTRODUCTION
\section{Introduction}
\subsection{The Problem Field}
\frame {
    \frametitle{The Problem Field}
    \begin{itemize}
    \item Need for support of business processes
    \item Hard coded business processes in information systems
    \end{itemize}

    \begin{itemize}
    \item Expensive change of processes and actions
    \item Software behavior out of synchronization % with real processes
    \item Software must be "tricked out" % to get the results which are needed
    \end{itemize}
}

\subsection{What is AMREP?}
\frame {
    \frametitle{What is an Activity Model Runtime Engine for Python?}
    \begin{itemize}
    \item Activity
        \begin{itemize}
        \item Simplified UML2 activity specification
        \item Comparable with BPMN
        \end{itemize}
    \item Model
        \begin{itemize}
        \item Simplified representation of a system
        \item Metamodel based
        \end{itemize}
    \item Runtime Engine
        \begin{itemize}
        \item Model interpreter
        \item Reads and evaluates the model at runtime
        \item Allows model changing at runtime
        \end{itemize}
    \item Python
        \begin{itemize}
        \item Programming language
        \item Dynamic, interpreted and object oriented
        % \item 7th most popular programming language\footnote{\scriptsize{Regarding to TIOBE Software BV}}
        % http://www.tiobe.com/index.php/content/paperinfo/tpci/index.html
        % http://langpop.com/
        \end{itemize}
    \end{itemize}
}


\subsection{AMREP and the Metamodel}
\frame {
    \frametitle{AMREP and the Metamodel}
    \begin{columns}[c]
    \column{2in}
    \begin{itemize}
    \item A metamodel defines the language of a model
    \item AMREP implements a subset of UML2
    \end{itemize}

    \column{2in}
    \includegraphics[scale=0.7]{resources/metalevel}
    \end{columns}

}

% IMPLEMENTATION METAMODEL
\section{The AMREP Metamodel}
\subsection{Metamodel Detail} % Detail from the Metamodel
\frame {
    \includegraphics[scale=0.48]{resources/activities-metamodel-simplified}
}

\subsection{Model Example in UML2 Notation}  % Activity Model Example in UML2 Notation
\frame {
    \includegraphics[scale=0.6]{resources/activity-example}
}


\subsection{Validation of the Model}
\frame {
    \frametitle{Validation of the Model}
    \begin{itemize}
    \item Checks conformity with metamodel rules
    \item Check after model creation or model change
    \item Examples:
        \begin{itemize}
        \item An InitialNode has no incoming edges
        \item An Edge must have source and target node set
        \item A DecisionNode must have one incoming edge and at least one outgoing edge
        \end{itemize}
    \end{itemize}
}

% IMPLEMENTATION RUNTIME ENGINE
\section{The AMREP Runtime Engine}
\subsection{Model Execution}
\frame {
    \frametitle{Model Execution}
    \begin{enumerate}
    \item Instantiation of the runtime class
        \begin{itemize}
        \item Parameter: Model
        \end{itemize}
    \item Calling the start-method
        \begin{itemize}
        \item Optional parameter: Any data necessary for model execution
        \end{itemize}
    \item Calling the next-method
        \begin{itemize}
        \item Called until activity is finished
        \end{itemize}
    \end{enumerate}

    \begin{itemize}
    \item Token based runtime engine
        \begin{itemize}
        \item Concept from Petri Nets
        \item Tokens represent activity execution status
        \item Tokens can hold data
        \item Tokens are managed in a token pool
        \end{itemize}
    \end{itemize}
}

\subsection{The Runtime Start Process}
\frame {
    \includegraphics[scale=0.7]{resources/runtime-start-simplified}
}

\subsection{The Runtime Next Process}
\frame {
    \includegraphics[scale=0.6]{resources/runtime-next-simplified}
}

\subsection{Executing Actions}
\frame {
    \frametitle{Executing Actions}
    \begin{itemize}
    \item Action node delegates execution to %to a external implementation of the action to be performed
    \item External implementation which
    \item Receives runtime data and
    \item Returns eventually manipulated data
    \end{itemize}
}


% CONCLUSION
\section{Conclusion}
\subsection{Use Cases for AMREP}
\frame {
    \frametitle{Use Cases for AMREP}
    \begin{itemize}
    \item Supporting business processes
    \item Document workflow support % beyond state machines
    \item Web form flow
    \item Industrial automation
    \item Visual programming
    \item ...
    \end{itemize}
}

\subsection{Potentials for Further Development}
\frame {
    \frametitle{Potentials for Further Development}
    \begin{itemize}
    \item Alignment of metamodel to UML2 specification % in particular some elements which are not implemented like ExectionHandler, InteruptibleActivityRegion, SendSignalAction, AcceptEventAction, ActivityPartition
    \item Improvement of XMI import
    \item Support of asynchronous action execution
    \item Visualization of the token status
    \end{itemize}
}

\subsection{Outcome}
\frame {
    \frametitle{Outcome}
    \begin{itemize}
    \item Unique framework for python
    \item Extensible architecture
    \item Usable prototype
    \item Supports predefined use case % May be adapt to support more use cases
    \item Functionality proofed by tests
    \item Free Software / Lesser Gnu Public License
    \end{itemize}
}

\frame {
    \center{\huge{Thank you for your attention}}
}


\end{document}



%\section{KAPITEL}
%\subsection{TITLE}
%\frame {
%    \frametitle{TITLE}
%    \begin{itemize}
%    \item
%    \end{itemize}
%}






% END OF DOCUMENT
