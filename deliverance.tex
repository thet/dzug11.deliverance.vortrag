Website Theming mit Deliverance und XDV
=======================================

% TODO: graphik - beispiel - 4 Folien

Inhalt des Vortrags

Websitegestaltung ist essentiell
    Eine der ersten Aufgaben bei Erstellung von Websites

Das Problemfeld: Webdesign aktuell

    Notwendiges Wissen
        HTML
        CSS
        Javascript

        Deploymentprozess
            SVN, GIT, ...
        Templatemechanismus des Frameworks
            ZPT, Smarty, PHP, JSP, ...

        Unterschiedliche Systeme - Unterschiedliche Techniken
        Wissen über Zielsystem muss vorhanden sein

    Workflow
        Designteam: HTML Dummy Template
        Entwicklungsteam - Zusatzaufgaben: Integration in Zielsystem, Veröffentlichung.
        Wasserfallmodell - Nachbesserungen für Designteam ist nachfolgend mühsam. Entwicklungsteam muss für Designänderungen immer kontaktiert werden.

    % TODO: graphik - Beispiel Plone


Was ist Deliverance?
    Theming Engine
        Basiert auf Konzepten von XSLT - Ursprünglich war es ein XSLT compiler
        Einfache Regeln beschreiben Transformation
        Basiert auf einer Idee von Paul Everitt aus dem Jahr 2005

    Middlewaretool, Proxy % TODO: Graphik Systemskizze Cachingserver, Webserver, Deliverance, CMS System

    HTML Transformation durch Mapping von Elementen % TODO: graphik ~ mapping
        Inhalte aus CMS System werden in Platzhalter von HTML Template gemappt
        Ähnlich XSL Transformationen
        Aus Content, Theme und XML Datei mit Regelanweisungen wird neues Dokument generiert
        Theme und Content müssen werden dabei nicht verändert
        Valides HTML und eindeutig selektierbare (CSS/XPath) Elemente notwendig

    Trennung der Präsentation von Inhalt und Logik
        Viel saubere Trennung als mit Templatesprachen meistens umgesetzt

###########

    Notwendiges Wissen
        HTML
        CSS
        Tools wie Firebug
        Aufbau der XML Rules Datei

    Unterschiedliche Systeme mit einer Software themen
        Einheitliches Layout mit einem HTML Theme möglich
        Alle Resourcen (Content, Theme) können per HTTP geholt werden
        Einfache Web-Mashups

Deliverance Regeln
    drop
        <drop content="SELEKTOR" /> | <drop theme="SELEKTOR" />
    replace
        <replace content="SELEKTOR" theme="SELEKTOR" />
    prepend
        <prepend content="SELEKTOR" theme="SELEKTOR" />
    append
        <prepend content="SELEKTOR" theme="SELEKTOR" />

Deliverance Selektoren
    CSS 3 Selektoren
        #content, .header, div[href~='.kss']
    XPath
        /html/head/title
        /html/body/div[1]/a[3]
    Selektor Typen
        element
        children
        attribute
        ...
    % Ruleset, Theme, Proxy, Rules, Classes, Matches


Aufbau der Deliverance Konfiguration
    % easy rules.xml beispiel


Deliverance Deployment
    Proxy Server
    WSGI Middleware
    Transformationen werden bei jedem Zugriff neu erstellt
    Caching Proxy von Vorteil


Grenzen von Deliverance
    Keine Schleifen-Konstrukte.
        Bsp: Variable Anzahl von <li> Elementen aus Menü kann nicht sinnvoll in ein anderes Konstrukt im Theme gemappt werden.
    Kleinteilige Layoutarbeiten können schneller in Templatesprache umgesetzt werden
        Bsp: Layoutänderungen in Plone im content-core. Anordnung von Artikel-Elementen, Folder Listings, etc.

    <---> XDV: Unterstützt Inline-XSL

Warum trotzdem Deliverance?

Der Deliverance Workflow
    Designteam
        Erstellt HTML Theme und CSS Dateien, ev. JS
        Kann rules.xml editieren
        Bei Zugriff auf Entwicklungs- bzw Produktivserver keine Interaktion mit Entwicklungsteam notwendig

    Entwicklungsteam
        Umsetzen von kleinteiligen Layoutarbeiten (content-core)
        Erstellung rules.xml

    System Admin Team
        Bereitstellen der Infrastruktur (Proxy Server)


Case Study: Grüne Akademie Steiermark
    % TODO: Systemskizze: Designer, Dropbox, Entwickler, Entwicklungsserver, Produktivserver, GIT
    % TODO: Website ungethemed - Website gethemed - Website eingeloggt, gethemed


% TODO TODO
XDV
    Unterschiede zu Deliverance
        XSLT compiler
        Regeln
        Deployment
        Vorteile
        Nachteile

% TODO TODO
Performance und Skalierbarkeit
    XDV vs Deliverance

% TODO TODO
Zukunft
    Banjo - Web based Deliverance Rule Editing


#
